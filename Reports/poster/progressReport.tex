\documentclass[12pt]{article}

\usepackage[margin=1in]{geometry}
\setlength{\parindent}{0pt}

\begin{document}
\title{Poster Preparation}
\date{}
\maketitle
\section{Problem Statement}
\subsection{Goal}
Our project was to develop a code base for future Arduino-based prototyping for
an existing transmission control system. CAN compatibility is required for 
integration with the existing testbed and other development tools. 

\subsection{Requirements}
The delivered code base should consist of independent modules that have no external
dependencies. The operations accomplished by the modules should be complete and 
require minimal customization.

\subsection{Objectives}
\begin{itemize}
	\item	Implement sensor reading for Arduino
	\item	Transmit useful readings by CAN protocol
	\item	Implement data logging on a receiving CAN node
	\item	Integrate project-specific sensor into existing network
\end{itemize}

\subsection{Applications}
\begin{itemize}
	\item	Integration of digital and analog sensors with existing testbed
	\item	Rapid prototyping with modules from code base
\end{itemize}

\section{Work accomplished}
\begin{itemize}
	\item	Integration of digital sensor with Arudino-based controller
	\item	CAN network integration of Arduino
		\begin{itemize}
			\item	Standard CAN integration
			\item	Implementation of J1939 variant
		\end{itemize}
	\item 	Integration of sensor into CAN network
		\begin{itemize}
			\item	Digital sensor
			\item	Analog sensor with external circuitry
		\end{itemize}
	\item	Real time data logging of transmitted network information
	\item	System validation using industry tools and typical
		application parameters
	\item	Out-of-the-box readiness of modules
\end{itemize}

\section{Conclusions}
\begin{itemize}
	\item	Agile concepts from software development useful
		for various environments
	\item	Important part of project was selecting tools and
		systems, not developing from scratch
	\item	Team work not comparable to previous experience and
		required adaptive approach
\end{itemize}	
\end{document}
